% Options for packages loaded elsewhere
\PassOptionsToPackage{unicode}{hyperref}
\PassOptionsToPackage{hyphens}{url}
%
\documentclass[
]{article}
\usepackage{amsmath,amssymb}
\usepackage{lmodern}
\usepackage{ifxetex,ifluatex}
\ifnum 0\ifxetex 1\fi\ifluatex 1\fi=0 % if pdftex
  \usepackage[T1]{fontenc}
  \usepackage[utf8]{inputenc}
  \usepackage{textcomp} % provide euro and other symbols
\else % if luatex or xetex
  \usepackage{unicode-math}
  \defaultfontfeatures{Scale=MatchLowercase}
  \defaultfontfeatures[\rmfamily]{Ligatures=TeX,Scale=1}
\fi
% Use upquote if available, for straight quotes in verbatim environments
\IfFileExists{upquote.sty}{\usepackage{upquote}}{}
\IfFileExists{microtype.sty}{% use microtype if available
  \usepackage[]{microtype}
  \UseMicrotypeSet[protrusion]{basicmath} % disable protrusion for tt fonts
}{}
\makeatletter
\@ifundefined{KOMAClassName}{% if non-KOMA class
  \IfFileExists{parskip.sty}{%
    \usepackage{parskip}
  }{% else
    \setlength{\parindent}{0pt}
    \setlength{\parskip}{6pt plus 2pt minus 1pt}}
}{% if KOMA class
  \KOMAoptions{parskip=half}}
\makeatother
\usepackage{xcolor}
\IfFileExists{xurl.sty}{\usepackage{xurl}}{} % add URL line breaks if available
\IfFileExists{bookmark.sty}{\usepackage{bookmark}}{\usepackage{hyperref}}
\hypersetup{
  pdftitle={Genes and Geography in Europe},
  hidelinks,
  pdfcreator={LaTeX via pandoc}}
\urlstyle{same} % disable monospaced font for URLs
\usepackage[margin=1in]{geometry}
\usepackage{color}
\usepackage{fancyvrb}
\newcommand{\VerbBar}{|}
\newcommand{\VERB}{\Verb[commandchars=\\\{\}]}
\DefineVerbatimEnvironment{Highlighting}{Verbatim}{commandchars=\\\{\}}
% Add ',fontsize=\small' for more characters per line
\usepackage{framed}
\definecolor{shadecolor}{RGB}{248,248,248}
\newenvironment{Shaded}{\begin{snugshade}}{\end{snugshade}}
\newcommand{\AlertTok}[1]{\textcolor[rgb]{0.94,0.16,0.16}{#1}}
\newcommand{\AnnotationTok}[1]{\textcolor[rgb]{0.56,0.35,0.01}{\textbf{\textit{#1}}}}
\newcommand{\AttributeTok}[1]{\textcolor[rgb]{0.77,0.63,0.00}{#1}}
\newcommand{\BaseNTok}[1]{\textcolor[rgb]{0.00,0.00,0.81}{#1}}
\newcommand{\BuiltInTok}[1]{#1}
\newcommand{\CharTok}[1]{\textcolor[rgb]{0.31,0.60,0.02}{#1}}
\newcommand{\CommentTok}[1]{\textcolor[rgb]{0.56,0.35,0.01}{\textit{#1}}}
\newcommand{\CommentVarTok}[1]{\textcolor[rgb]{0.56,0.35,0.01}{\textbf{\textit{#1}}}}
\newcommand{\ConstantTok}[1]{\textcolor[rgb]{0.00,0.00,0.00}{#1}}
\newcommand{\ControlFlowTok}[1]{\textcolor[rgb]{0.13,0.29,0.53}{\textbf{#1}}}
\newcommand{\DataTypeTok}[1]{\textcolor[rgb]{0.13,0.29,0.53}{#1}}
\newcommand{\DecValTok}[1]{\textcolor[rgb]{0.00,0.00,0.81}{#1}}
\newcommand{\DocumentationTok}[1]{\textcolor[rgb]{0.56,0.35,0.01}{\textbf{\textit{#1}}}}
\newcommand{\ErrorTok}[1]{\textcolor[rgb]{0.64,0.00,0.00}{\textbf{#1}}}
\newcommand{\ExtensionTok}[1]{#1}
\newcommand{\FloatTok}[1]{\textcolor[rgb]{0.00,0.00,0.81}{#1}}
\newcommand{\FunctionTok}[1]{\textcolor[rgb]{0.00,0.00,0.00}{#1}}
\newcommand{\ImportTok}[1]{#1}
\newcommand{\InformationTok}[1]{\textcolor[rgb]{0.56,0.35,0.01}{\textbf{\textit{#1}}}}
\newcommand{\KeywordTok}[1]{\textcolor[rgb]{0.13,0.29,0.53}{\textbf{#1}}}
\newcommand{\NormalTok}[1]{#1}
\newcommand{\OperatorTok}[1]{\textcolor[rgb]{0.81,0.36,0.00}{\textbf{#1}}}
\newcommand{\OtherTok}[1]{\textcolor[rgb]{0.56,0.35,0.01}{#1}}
\newcommand{\PreprocessorTok}[1]{\textcolor[rgb]{0.56,0.35,0.01}{\textit{#1}}}
\newcommand{\RegionMarkerTok}[1]{#1}
\newcommand{\SpecialCharTok}[1]{\textcolor[rgb]{0.00,0.00,0.00}{#1}}
\newcommand{\SpecialStringTok}[1]{\textcolor[rgb]{0.31,0.60,0.02}{#1}}
\newcommand{\StringTok}[1]{\textcolor[rgb]{0.31,0.60,0.02}{#1}}
\newcommand{\VariableTok}[1]{\textcolor[rgb]{0.00,0.00,0.00}{#1}}
\newcommand{\VerbatimStringTok}[1]{\textcolor[rgb]{0.31,0.60,0.02}{#1}}
\newcommand{\WarningTok}[1]{\textcolor[rgb]{0.56,0.35,0.01}{\textbf{\textit{#1}}}}
\usepackage{longtable,booktabs,array}
\usepackage{calc} % for calculating minipage widths
% Correct order of tables after \paragraph or \subparagraph
\usepackage{etoolbox}
\makeatletter
\patchcmd\longtable{\par}{\if@noskipsec\mbox{}\fi\par}{}{}
\makeatother
% Allow footnotes in longtable head/foot
\IfFileExists{footnotehyper.sty}{\usepackage{footnotehyper}}{\usepackage{footnote}}
\makesavenoteenv{longtable}
\usepackage{graphicx}
\makeatletter
\def\maxwidth{\ifdim\Gin@nat@width>\linewidth\linewidth\else\Gin@nat@width\fi}
\def\maxheight{\ifdim\Gin@nat@height>\textheight\textheight\else\Gin@nat@height\fi}
\makeatother
% Scale images if necessary, so that they will not overflow the page
% margins by default, and it is still possible to overwrite the defaults
% using explicit options in \includegraphics[width, height, ...]{}
\setkeys{Gin}{width=\maxwidth,height=\maxheight,keepaspectratio}
% Set default figure placement to htbp
\makeatletter
\def\fps@figure{htbp}
\makeatother
\setlength{\emergencystretch}{3em} % prevent overfull lines
\providecommand{\tightlist}{%
  \setlength{\itemsep}{0pt}\setlength{\parskip}{0pt}}
\setcounter{secnumdepth}{-\maxdimen} % remove section numbering
\ifluatex
  \usepackage{selnolig}  % disable illegal ligatures
\fi

\title{Genes and Geography in Europe}
\author{}
\date{\vspace{-2.5em}}

\begin{document}
\maketitle

\hypertarget{preparing-the-dataset}{%
\subsection{Preparing the dataset}\label{preparing-the-dataset}}

First we'll set up the environment. We need the \texttt{tidyverse}
packages, as well as the \texttt{vegan} package to perfrorm the
procrustes analysis

\begin{Shaded}
\begin{Highlighting}[]
\FunctionTok{library}\NormalTok{(tidyverse)}
\end{Highlighting}
\end{Shaded}

\begin{verbatim}
## -- Attaching packages ------------------------------------------------------------------------------------- tidyverse 1.3.0 --
\end{verbatim}

\begin{verbatim}
## v ggplot2 3.3.0     v purrr   0.3.3
## v tibble  2.1.3     v dplyr   0.8.5
## v tidyr   1.0.2     v stringr 1.4.0
## v readr   1.3.1     v forcats 0.5.0
\end{verbatim}

\begin{verbatim}
## -- Conflicts ---------------------------------------------------------------------------------------- tidyverse_conflicts() --
## x dplyr::filter() masks stats::filter()
## x dplyr::lag()    masks stats::lag()
\end{verbatim}

\begin{Shaded}
\begin{Highlighting}[]
\FunctionTok{library}\NormalTok{(vegan)}
\end{Highlighting}
\end{Shaded}

\begin{verbatim}
## Loading required package: permute
\end{verbatim}

\begin{verbatim}
## Loading required package: lattice
\end{verbatim}

\begin{verbatim}
## This is vegan 2.5-7
\end{verbatim}

Let's read in the data

\begin{Shaded}
\begin{Highlighting}[]
\NormalTok{pca }\OtherTok{\textless{}{-}} \FunctionTok{read\_tsv}\NormalTok{(}\StringTok{"datasets/novembre\_2008\_pca.tsv"}\NormalTok{)}
\end{Highlighting}
\end{Shaded}

\begin{verbatim}
## Parsed with column specification:
## cols(
##   sampleId = col_character(),
##   country = col_character(),
##   alabels = col_character(),
##   longitude = col_double(),
##   latitude = col_double(),
##   PC1 = col_double(),
##   PC2 = col_double()
## )
\end{verbatim}

\begin{Shaded}
\begin{Highlighting}[]
\NormalTok{colors }\OtherTok{\textless{}{-}} \FunctionTok{read\_tsv}\NormalTok{(}\StringTok{"datasets/novembre\_2008\_colours.tsv"}\NormalTok{, }\AttributeTok{col\_names =} \FunctionTok{c}\NormalTok{(}\StringTok{"country"}\NormalTok{, }\StringTok{"color"}\NormalTok{))}
\end{Highlighting}
\end{Shaded}

\begin{verbatim}
## Parsed with column specification:
## cols(
##   country = col_character(),
##   color = col_character()
## )
\end{verbatim}

The variable \texttt{pca} contains the results of the principal
component analysis of 1,387 European individuals from Novembre et al
2008. Let's have a look at the first few lines of the dataset:

\begin{longtable}[]{@{}lllrrrr@{}}
\toprule
sampleId & country & alabels & longitude & latitude & PC1 & PC2 \\
\midrule
\endhead
ind\_474 & Croatia & HR & 16.10000 & 45.32000 & 0.0111 & -0.0460 \\
ind\_660 & Serbia and Montenegro & YG & 20.61572 & 43.94949 & -0.0300 &
-0.0443 \\
ind\_1890 & Czech Republic & CZ & 15.37698 & 49.73661 & 0.0137 &
-0.0444 \\
ind\_1923 & Bosnia and Herzegovina & BA & 17.86202 & 44.17588 & 0.0028 &
-0.0493 \\
ind\_2083 & Serbia and Montenegro & YG & 20.61572 & 43.94949 & -0.0031 &
-0.0416 \\
ind\_2733 & Serbia and Montenegro & YG & 20.61572 & 43.94949 & -0.0026 &
-0.0424 \\
\bottomrule
\end{longtable}

The color palette used for the countries is stored as a table in the
variable \texttt{colors}:

\begin{longtable}[]{@{}ll@{}}
\toprule
country & color \\
\midrule
\endhead
Germany & gold1 \\
Netherlands & orange \\
Austria & tan3 \\
Luxembourg & yellow \\
Czech Republic & yellowgreen \\
Hungary & gold2 \\
\bottomrule
\end{longtable}

\hypertarget{rotating-the-pca}{%
\subsection{Rotating the PCA}\label{rotating-the-pca}}

In order to find the rotation for the PCA that aligns best with the
geographic location of the samples, we will use the
\texttt{procrustes()} function from the package \texttt{procrustes}. We
first need to convert both the principal components and geographic
locations of the samples into matrix format.

\begin{Shaded}
\begin{Highlighting}[]
\NormalTok{loc }\OtherTok{\textless{}{-}}\NormalTok{ pca }\SpecialCharTok{\%\textgreater{}\%}
    \FunctionTok{select}\NormalTok{(longitude, latitude) }\SpecialCharTok{\%\textgreater{}\%}
    \FunctionTok{as.matrix}\NormalTok{()}

\NormalTok{pca1 }\OtherTok{\textless{}{-}}\NormalTok{ pca }\SpecialCharTok{\%\textgreater{}\%}
    \FunctionTok{select}\NormalTok{(PC1, PC2) }\SpecialCharTok{\%\textgreater{}\%}
    \FunctionTok{as.matrix}\NormalTok{()}
\end{Highlighting}
\end{Shaded}

Now we carry out the procrustes analysis to find the optimal rotation

\begin{Shaded}
\begin{Highlighting}[]
\NormalTok{rot }\OtherTok{\textless{}{-}} \FunctionTok{procrustes}\NormalTok{(loc, pca1)}
\end{Highlighting}
\end{Shaded}

We can examine the result by using base R \texttt{plot} function on the
resulting object

\begin{Shaded}
\begin{Highlighting}[]
\FunctionTok{plot}\NormalTok{(rot)}
\end{Highlighting}
\end{Shaded}

\includegraphics{10_analyze_novembre2008_files/figure-latex/procrustes3-1.pdf}

The rotated PC coordinates can be extracted from the \texttt{Yrot}
element of the result object. Here we add two new columns for the
rotated PCs to our \texttt{pca} object

\begin{Shaded}
\begin{Highlighting}[]
\NormalTok{pca }\OtherTok{\textless{}{-}}\NormalTok{ pca }\SpecialCharTok{\%\textgreater{}\%}
    \FunctionTok{mutate}\NormalTok{(}\AttributeTok{PC1\_rot =}\NormalTok{ rot}\SpecialCharTok{$}\NormalTok{Yrot[,}\DecValTok{1}\NormalTok{], }\AttributeTok{PC2\_rot =}\NormalTok{ rot}\SpecialCharTok{$}\NormalTok{Yrot[,}\DecValTok{2}\NormalTok{])}
\end{Highlighting}
\end{Shaded}

\hypertarget{recreating-figure-1-of-novembre-et-al-2008}{%
\subsection{Recreating Figure 1 of Novembre et al
2008}\label{recreating-figure-1-of-novembre-et-al-2008}}

With the rotated PCA results calculated, everything is in place for
creating the figure. First we set up the color scale for the countries

\begin{Shaded}
\begin{Highlighting}[]
\NormalTok{colorscale }\OtherTok{\textless{}{-}}\NormalTok{ colors}\SpecialCharTok{$}\NormalTok{color}
\FunctionTok{names}\NormalTok{(colorscale) }\OtherTok{\textless{}{-}}\NormalTok{ colors}\SpecialCharTok{$}\NormalTok{country}
\end{Highlighting}
\end{Shaded}

Next, we calculate the median PC values for each country

\begin{Shaded}
\begin{Highlighting}[]
\NormalTok{pca\_summary }\OtherTok{\textless{}{-}}\NormalTok{ pca }\SpecialCharTok{\%\textgreater{}\%}
    \FunctionTok{group\_by}\NormalTok{(country, alabels) }\SpecialCharTok{\%\textgreater{}\%}
    \FunctionTok{summarise\_at}\NormalTok{(}\FunctionTok{c}\NormalTok{(}\StringTok{"PC1\_rot"}\NormalTok{, }\StringTok{"PC2\_rot"}\NormalTok{), median)}
\end{Highlighting}
\end{Shaded}

To add rotated axes lines, we take the coordinates of the four points at
the end of the axis ranges, and rotate them into the new coordinate
space using the \texttt{predict} function on the procrustes output
object

\begin{Shaded}
\begin{Highlighting}[]
\NormalTok{lim\_pc1 }\OtherTok{\textless{}{-}} \FunctionTok{extendrange}\NormalTok{(pca}\SpecialCharTok{$}\NormalTok{PC1)}
\NormalTok{lim\_pc2 }\OtherTok{\textless{}{-}} \FunctionTok{extendrange}\NormalTok{(pca}\SpecialCharTok{$}\NormalTok{PC2)}

\NormalTok{axis }\OtherTok{\textless{}{-}} \FunctionTok{rbind}\NormalTok{(}\FunctionTok{c}\NormalTok{(lim\_pc1[}\DecValTok{1}\NormalTok{], }\DecValTok{0}\NormalTok{), }\FunctionTok{c}\NormalTok{(lim\_pc1[}\DecValTok{2}\NormalTok{], }\DecValTok{0}\NormalTok{), }\FunctionTok{c}\NormalTok{(}\DecValTok{0}\NormalTok{, lim\_pc2[}\DecValTok{1}\NormalTok{]), }\FunctionTok{c}\NormalTok{(}\DecValTok{0}\NormalTok{, lim\_pc2[}\DecValTok{2}\NormalTok{]))}
\NormalTok{axis\_rot }\OtherTok{\textless{}{-}} \FunctionTok{predict}\NormalTok{(rot, axis, }\AttributeTok{truemean =} \ConstantTok{FALSE}\NormalTok{)}
\NormalTok{axis\_rot\_df  }\OtherTok{\textless{}{-}} \FunctionTok{tibble}\NormalTok{(}\AttributeTok{x =}\NormalTok{ axis\_rot[}\FunctionTok{c}\NormalTok{(}\DecValTok{1}\NormalTok{, }\DecValTok{3}\NormalTok{),}\DecValTok{1}\NormalTok{], }\AttributeTok{xend =}\NormalTok{ axis\_rot[}\FunctionTok{c}\NormalTok{(}\DecValTok{2}\NormalTok{, }\DecValTok{4}\NormalTok{),}\DecValTok{1}\NormalTok{], }\AttributeTok{y =}\NormalTok{ axis\_rot[}\FunctionTok{c}\NormalTok{(}\DecValTok{1}\NormalTok{, }\DecValTok{3}\NormalTok{),}\DecValTok{2}\NormalTok{], }\AttributeTok{yend =}\NormalTok{ axis\_rot[}\FunctionTok{c}\NormalTok{(}\DecValTok{2}\NormalTok{, }\DecValTok{4}\NormalTok{), }\DecValTok{2}\NormalTok{])}
\end{Highlighting}
\end{Shaded}

Similarly, we calculate rotated coordinates for two axis labels, in a
position shifted towards the origin from the axis end point

\begin{Shaded}
\begin{Highlighting}[]
\NormalTok{labels }\OtherTok{\textless{}{-}} \FunctionTok{rbind}\NormalTok{(}\FunctionTok{c}\NormalTok{(lim\_pc1[}\DecValTok{2}\NormalTok{] }\SpecialCharTok{*} \FloatTok{0.85}\NormalTok{, }\DecValTok{0}\NormalTok{), }\FunctionTok{c}\NormalTok{(}\DecValTok{0}\NormalTok{, lim\_pc2[}\DecValTok{2}\NormalTok{] }\SpecialCharTok{*} \FloatTok{0.85}\NormalTok{))}
\NormalTok{labels\_rot }\OtherTok{\textless{}{-}} \FunctionTok{predict}\NormalTok{(rot, labels, }\AttributeTok{truemean =} \ConstantTok{FALSE}\NormalTok{) }\SpecialCharTok{\%\textgreater{}\%}
    \FunctionTok{as\_tibble}\NormalTok{() }\SpecialCharTok{\%\textgreater{}\%}
    \FunctionTok{mutate}\NormalTok{(}\AttributeTok{labels =} \FunctionTok{c}\NormalTok{(}\StringTok{"PC1"}\NormalTok{, }\StringTok{"PC2"}\NormalTok{))}
\end{Highlighting}
\end{Shaded}

\begin{verbatim}
## Warning: `as_tibble.matrix()` requires a matrix with column names or a `.name_repair` argument. Using compatibility `.name_repair`.
## This warning is displayed once per session.
\end{verbatim}

\begin{Shaded}
\begin{Highlighting}[]
\FunctionTok{colnames}\NormalTok{(labels\_rot)[}\DecValTok{1}\SpecialCharTok{:}\DecValTok{2}\NormalTok{] }\OtherTok{\textless{}{-}} \FunctionTok{c}\NormalTok{(}\StringTok{"PC1\_rot"}\NormalTok{, }\StringTok{"PC2\_rot"}\NormalTok{)}
\end{Highlighting}
\end{Shaded}

Finally, we need to calculate the rotation angle from the rotation
matrix for the axis label text

\begin{Shaded}
\begin{Highlighting}[]
\NormalTok{theta1 }\OtherTok{\textless{}{-}} \FunctionTok{acos}\NormalTok{(rot}\SpecialCharTok{$}\NormalTok{rotation[}\DecValTok{1}\NormalTok{,}\DecValTok{1}\NormalTok{]) }\SpecialCharTok{*} \DecValTok{180} \SpecialCharTok{/}\NormalTok{ pi}
\end{Highlighting}
\end{Shaded}

With all objects in place we can now build the final plot

\begin{Shaded}
\begin{Highlighting}[]
\NormalTok{p }\OtherTok{\textless{}{-}} \FunctionTok{ggplot}\NormalTok{(pca, }\FunctionTok{aes}\NormalTok{(}\AttributeTok{x =}\NormalTok{ PC1\_rot, }\AttributeTok{y =}\NormalTok{ PC2\_rot))}
\NormalTok{p }\SpecialCharTok{+}
    \FunctionTok{geom\_segment}\NormalTok{(}\FunctionTok{aes}\NormalTok{(}\AttributeTok{x =}\NormalTok{ x, }\AttributeTok{xend =}\NormalTok{ xend, }\AttributeTok{y =}\NormalTok{ y, }\AttributeTok{yend =}\NormalTok{ yend), }\AttributeTok{size =} \FloatTok{0.25}\NormalTok{, }\AttributeTok{arrow =} \FunctionTok{arrow}\NormalTok{(}\AttributeTok{ends =} \StringTok{"both"}\NormalTok{, }\AttributeTok{angle =} \DecValTok{10}\NormalTok{, }\AttributeTok{type =} \StringTok{"closed"}\NormalTok{, }\AttributeTok{length =} \FunctionTok{unit}\NormalTok{(}\FloatTok{0.02}\NormalTok{, }\StringTok{"npc"}\NormalTok{)), }\AttributeTok{data =}\NormalTok{ axis\_rot\_df) }\SpecialCharTok{+}
    \FunctionTok{geom\_point}\NormalTok{(}\AttributeTok{size =} \DecValTok{10}\NormalTok{, }\AttributeTok{colour =} \StringTok{"white"}\NormalTok{, }\AttributeTok{data =}\NormalTok{ labels\_rot) }\SpecialCharTok{+}
    \FunctionTok{geom\_text}\NormalTok{(}\FunctionTok{aes}\NormalTok{(}\AttributeTok{label =}\NormalTok{ labels), }\AttributeTok{angle =} \FunctionTok{c}\NormalTok{(theta1 }\SpecialCharTok{+} \DecValTok{180}\NormalTok{, theta1 }\SpecialCharTok{+} \DecValTok{270}\NormalTok{), }\AttributeTok{data =}\NormalTok{ labels\_rot) }\SpecialCharTok{+}
    \FunctionTok{geom\_text}\NormalTok{(}\FunctionTok{aes}\NormalTok{(}\AttributeTok{label =}\NormalTok{ alabels, }\AttributeTok{colour =}\NormalTok{ country), }\AttributeTok{alpha =} \FloatTok{0.8}\NormalTok{, }\AttributeTok{size =} \DecValTok{2}\NormalTok{) }\SpecialCharTok{+}
    \FunctionTok{geom\_point}\NormalTok{(}\FunctionTok{aes}\NormalTok{(}\AttributeTok{colour =}\NormalTok{ country), }\AttributeTok{size =} \DecValTok{7}\NormalTok{, }\AttributeTok{data =}\NormalTok{ pca\_summary) }\SpecialCharTok{+}
    \FunctionTok{geom\_text}\NormalTok{(}\FunctionTok{aes}\NormalTok{(}\AttributeTok{label =}\NormalTok{ alabels), }\AttributeTok{size =} \DecValTok{3}\NormalTok{, }\AttributeTok{data =}\NormalTok{ pca\_summary) }\SpecialCharTok{+}
    \FunctionTok{scale\_color\_manual}\NormalTok{(}\AttributeTok{name =} \StringTok{"Country"}\NormalTok{, }\AttributeTok{values =}\NormalTok{ colorscale) }\SpecialCharTok{+}
    \FunctionTok{coord\_equal}\NormalTok{() }\SpecialCharTok{+}
    \FunctionTok{theme\_void}\NormalTok{() }\SpecialCharTok{+}
    \FunctionTok{theme}\NormalTok{(}\AttributeTok{legend.position =} \StringTok{"none"}\NormalTok{)}
\end{Highlighting}
\end{Shaded}

\includegraphics{10_analyze_novembre2008_files/figure-latex/viz6-1.pdf}

\hypertarget{predicting-location-of-origin-from-the-pca-position}{%
\subsection{Predicting location of origin from the PCA
position}\label{predicting-location-of-origin-from-the-pca-position}}

In this last section, we will follow the paper and build a predictor for
an individual's geographic location from its PCA position, using linear
regression. As test case, we take all individuals of a country of
choice, remove them from the dataset and use the remaining individuals
to build the model. Then, we use the built model to predict the location
of the test individuals.

First we split our dataset into training and test data

\begin{Shaded}
\begin{Highlighting}[]
\NormalTok{test\_country }\OtherTok{\textless{}{-}} \StringTok{"Germany"}
\NormalTok{d\_train }\OtherTok{\textless{}{-}} \FunctionTok{filter}\NormalTok{(pca, country }\SpecialCharTok{!=}\NormalTok{ test\_country)}
\NormalTok{d\_test }\OtherTok{\textless{}{-}} \FunctionTok{filter}\NormalTok{(pca, country }\SpecialCharTok{==}\NormalTok{ test\_country)}
\end{Highlighting}
\end{Shaded}

Next, we carry out two linear regressions, modelling both longitude and
latitude as functions of the rotated PCs and including their interaction
term

\begin{Shaded}
\begin{Highlighting}[]
\NormalTok{lm\_long }\OtherTok{\textless{}{-}} \FunctionTok{lm}\NormalTok{(longitude }\SpecialCharTok{\textasciitilde{}}\NormalTok{ PC1\_rot }\SpecialCharTok{*}\NormalTok{ PC2\_rot, }\AttributeTok{data =}\NormalTok{ d\_train)}
\NormalTok{lm\_lat }\OtherTok{\textless{}{-}} \FunctionTok{lm}\NormalTok{(latitude }\SpecialCharTok{\textasciitilde{}}\NormalTok{ PC1\_rot }\SpecialCharTok{*}\NormalTok{ PC2\_rot, }\AttributeTok{data =}\NormalTok{ d\_train)}
\end{Highlighting}
\end{Shaded}

Now we use the \texttt{predict()} function to infer the predicted
geographic location for the test samples, and reshape the results into a
format useful for plotting

\begin{Shaded}
\begin{Highlighting}[]
\NormalTok{d1 }\OtherTok{\textless{}{-}} \FunctionTok{select}\NormalTok{(d\_test, sampleId, longitude, latitude) }\SpecialCharTok{\%\textgreater{}\%}
    \FunctionTok{mutate}\NormalTok{(}\AttributeTok{longitude\_pr =} \FunctionTok{predict}\NormalTok{(lm\_long, d\_test), }\AttributeTok{latitude\_pr =} \FunctionTok{predict}\NormalTok{(lm\_lat, d\_test), }\AttributeTok{longitude\_orig =}\NormalTok{ longitude, }\AttributeTok{latitude\_orig =}\NormalTok{ latitude) }\SpecialCharTok{\%\textgreater{}\%}
    \FunctionTok{select}\NormalTok{(}\SpecialCharTok{{-}}\NormalTok{longitude}\SpecialCharTok{:{-}}\NormalTok{latitude) }\SpecialCharTok{\%\textgreater{}\%}
    \FunctionTok{pivot\_longer}\NormalTok{(}\SpecialCharTok{{-}}\NormalTok{sampleId) }\SpecialCharTok{\%\textgreater{}\%}
    \FunctionTok{separate}\NormalTok{(name, }\AttributeTok{into =} \FunctionTok{c}\NormalTok{(}\StringTok{"variable"}\NormalTok{, }\StringTok{"group"}\NormalTok{)) }\SpecialCharTok{\%\textgreater{}\%}
    \FunctionTok{pivot\_wider}\NormalTok{(}\AttributeTok{names\_from =}\NormalTok{ variable, }\AttributeTok{values\_from =}\NormalTok{ value)}
\end{Highlighting}
\end{Shaded}

Finally, we can plot the inferred versus actual sample locations using
the basic map plotting functionality in \texttt{ggplot2}

\begin{Shaded}
\begin{Highlighting}[]
\NormalTok{w }\OtherTok{\textless{}{-}} \FunctionTok{map\_data}\NormalTok{(}\StringTok{"world"}\NormalTok{)}

\NormalTok{lim\_long }\OtherTok{\textless{}{-}} \FunctionTok{extendrange}\NormalTok{(pca}\SpecialCharTok{$}\NormalTok{longitude)}
\NormalTok{lim\_lat }\OtherTok{\textless{}{-}} \FunctionTok{extendrange}\NormalTok{(pca}\SpecialCharTok{$}\NormalTok{latitude)}

\NormalTok{pal }\OtherTok{\textless{}{-}} \FunctionTok{c}\NormalTok{(}\StringTok{"black"}\NormalTok{, }\StringTok{"red"}\NormalTok{)}
\FunctionTok{names}\NormalTok{(pal) }\OtherTok{\textless{}{-}} \FunctionTok{c}\NormalTok{(}\StringTok{"orig"}\NormalTok{, }\StringTok{"pr"}\NormalTok{)}
\NormalTok{sh }\OtherTok{\textless{}{-}} \FunctionTok{c}\NormalTok{(}\DecValTok{16}\NormalTok{, }\DecValTok{4}\NormalTok{)}
\FunctionTok{names}\NormalTok{(sh) }\OtherTok{\textless{}{-}} \FunctionTok{c}\NormalTok{(}\StringTok{"orig"}\NormalTok{, }\StringTok{"pr"}\NormalTok{)}

\NormalTok{p }\OtherTok{\textless{}{-}} \FunctionTok{ggplot}\NormalTok{(w)}
\NormalTok{p }\SpecialCharTok{+}
    \FunctionTok{geom\_polygon}\NormalTok{(}\FunctionTok{aes}\NormalTok{(long, lat, }\AttributeTok{group =}\NormalTok{ group), }\AttributeTok{fill =} \StringTok{"white"}\NormalTok{, }\AttributeTok{color =} \StringTok{"grey"}\NormalTok{, }\AttributeTok{size =} \FloatTok{0.25}\NormalTok{) }\SpecialCharTok{+}
    \FunctionTok{geom\_point}\NormalTok{(}\FunctionTok{aes}\NormalTok{(}\AttributeTok{x =}\NormalTok{ longitude, }\AttributeTok{y =}\NormalTok{ latitude, }\AttributeTok{colour =}\NormalTok{ group, }\AttributeTok{shape =}\NormalTok{ group), }\AttributeTok{size =} \DecValTok{2}\NormalTok{, }\AttributeTok{data =}\NormalTok{ d1) }\SpecialCharTok{+}
    \FunctionTok{scale\_color\_manual}\NormalTok{(}\AttributeTok{values =}\NormalTok{ pal) }\SpecialCharTok{+}
    \FunctionTok{scale\_shape\_manual}\NormalTok{(}\AttributeTok{values =}\NormalTok{ sh) }\SpecialCharTok{+}
    \FunctionTok{coord\_quickmap}\NormalTok{(}\AttributeTok{xlim =}\NormalTok{ lim\_long, }\AttributeTok{ylim =}\NormalTok{ lim\_lat)}
\end{Highlighting}
\end{Shaded}

\includegraphics{10_analyze_novembre2008_files/figure-latex/pred4-1.pdf}

\end{document}
